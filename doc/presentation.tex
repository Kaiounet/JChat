\documentclass{beamer}
\usetheme{Madrid}
\usecolortheme{rose}

\usepackage[utf8]{inputenc}
\usepackage[T1]{fontenc}
\usepackage[french]{babel}
\usepackage{listings}
\usepackage{xcolor}
\usepackage{tikz}
\usetikzlibrary{shapes,arrows,positioning}

% Code listing settings
\lstset{
    basicstyle=\ttfamily\tiny,
    keywordstyle=\color{blue}\bfseries,
    commentstyle=\color{green!60!black},
    stringstyle=\color{red},
    showspaces=false,
    showstringspaces=false,
    showtabs=false,
    frame=single,
    tabsize=2,
    breaklines=true,
    breakatwhitespace=false,
    literate=
      {é}{{\'e}}1
      {è}{{\`e}}1
      {ê}{{\^e}}1
      {à}{{\`a}}1
      {â}{{\^a}}1
      {î}{{\^i}}1
      {ô}{{\^o}}1
      {û}{{\^u}}1
      {ç}{{\c{c}}}1
      {É}{{\'E}}1
      {È}{{\`E}}1
      {Ê}{{\^E}}1
      {À}{{\`A}}1
}

\lstdefinestyle{bash}{
    language=bash,
    basicstyle=\ttfamily\scriptsize,
    backgroundcolor=\color{gray!10},
}

\lstdefinestyle{xml}{
    language=XML,
    basicstyle=\ttfamily\tiny,
    backgroundcolor=\color{gray!10},
}

\title{Configuration d'un Projet JavaFX depuis Zéro}
\subtitle{Utilisation de Maven et Make}
\author{Iliass \& Youness}
\date{\today}

\begin{document}

\frame{\titlepage}

\begin{frame}{Plan}
\tableofcontents
\end{frame}

\section{Prérequis}

\begin{frame}{Ce Dont Vous Avez Besoin}

\begin{block}{Logiciels Requis}
\begin{itemize}
    \item \textbf{Java JDK 11+} (recommandé : JDK 17)
    \item \textbf{Maven 3.6+}
    \item \textbf{Make} (optionnel mais recommandé)
    \item \textbf{Votre éditeur préféré} (Neovim, VS Code, IntelliJ, etc.)
\end{itemize}
\end{block}

\medskip % safer spacing in beamer than \vspace

\begin{block}{Vérifier l'Installation}
\hspace{0.5cm}java -version\\
\hspace{0.5cm}mvn -version\\
\hspace{0.5cm}make -version
\end{block}

\end{frame}

\begin{frame}{Pourquoi Cette Stack ?}
\begin{columns}
\column{0.47\textwidth}
\begin{block}{Maven}
\begin{itemize}
    \item Gestion des dépendances
    \item Structure du projet
    \item Cycle de vie du build
    \item Écosystème de plugins
\end{itemize}
\end{block}

\column{0.47\textwidth}
\begin{block}{Make}
\begin{itemize}
    \item Commandes simples
    \item Automatisation des tâches
    \item Multi-plateforme
    \item Convivial pour développeurs
\end{itemize}
\end{block}
\end{columns}

\vspace{0.5cm}
\centering
\textbf{Maven gère Java, Make gère la commodité !}
\end{frame}

\section{Création du Projet Maven}

\begin{frame}[fragile]{Étape 1 : Générer le Projet Maven}
\begin{block}{Exécuter Maven Archetype}
\begin{lstlisting}[style=bash]
mvn archetype:generate \
  -DgroupId=com.kaiounet \
  -DartifactId=javafx-app \
  -DarchetypeArtifactId=maven-archetype-quickstart \
  -DarchetypeVersion=1.4 \
  -DinteractiveMode=false
\end{lstlisting}
\end{block}

\begin{alertblock}{Ce Que Cela Fait}
Crée un projet Maven de base avec :
\begin{itemize}
    \item Structure de répertoires standard
    \item Basic \texttt{pom.xml}
    \item Exemple \texttt{App.java}
\end{itemize}
\end{alertblock}
\end{frame}

\begin{frame}[fragile]{Étape 2 : Naviguer vers le Projet}

\begin{block}{Structure du Projet Créée}
\begin{verbatim}
javafx-app/
|-- pom.xml
`-- src/
    |-- main/
    |   `-- java/
    |       `-- com/
    |           `-- kaiounet/
    |               `-- App.java
    `-- test/
        `-- java/
            `-- com/
                `-- kaiounet/
                    `-- AppTest.java
\end{verbatim}
\end{block}
\end{frame}

\section{Configuration de JavaFX}

\begin{frame}[fragile]{Étape 3 : Configurer pom.xml}
\begin{block}{Ouvrir pom.xml}
\begin{lstlisting}[style=bash]
nvim pom.xml
# ou
code pom.xml
# ou votre éditeur de choix
\end{lstlisting}
\end{block}

\vspace{0.3cm}

\begin{alertblock}{Configuration Clé Nécessaire}
\begin{enumerate}
    \item Version Java
    \item Dépendances JavaFX
    \item Plugin Maven JavaFX
\end{enumerate}
\end{alertblock}
\end{frame}

\begin{frame}[fragile]{pom.xml - Propriétés}
\begin{lstlisting}[style=xml]
<properties>
    <maven.compiler.source>17</maven.compiler.source>
    <maven.compiler.target>17</maven.compiler.target>
    <project.build.sourceEncoding>UTF-8</project.build.sourceEncoding>
    <javafx.version>21.0.1</javafx.version>
</properties>
\end{lstlisting}

\begin{itemize}
    \item \textbf{compiler.source/target} : Version Java
    \item \textbf{sourceEncoding} : UTF-8 pour tous les fichiers
    \item \textbf{javafx.version} : Version JavaFX à utiliser
\end{itemize}
\end{frame}

\begin{frame}[fragile]{pom.xml - Dépendances}
\begin{lstlisting}[style=xml]
<dependencies>
    <!-- JavaFX Controls -->
    <dependency>
        <groupId>org.openjfx</groupId>
        <artifactId>javafx-controls</artifactId>
        <version>${javafx.version}</version>
    </dependency>
    
    <!-- JavaFX FXML (optionnel) -->
    <dependency>
        <groupId>org.openjfx</groupId>
        <artifactId>javafx-fxml</artifactId>
        <version>${javafx.version}</version>
    </dependency>
</dependencies>
\end{lstlisting}

\textbf{javafx-controls} : Composants UI de base (Button, Label, etc.)
\end{frame}

\begin{frame}[fragile]{pom.xml - Plugin}
\begin{lstlisting}[style=xml]
<build>
    <plugins>
        <plugin>
            <groupId>org.openjfx</groupId>
            <artifactId>javafx-maven-plugin</artifactId>
            <version>0.0.8</version>
            <configuration>
                <mainClass>com.kaiounet.App</mainClass>
            </configuration>
        </plugin>
    </plugins>
</build>
\end{lstlisting}

\begin{alertblock}{Important}
\texttt{mainClass} doit correspondre à votre package et nom de classe !
\end{alertblock}
\end{frame}

\section{Création de l'Application JavaFX}

\begin{frame}[fragile]{Étape 4 : Créer Votre Application}
\begin{block}{Mettre à jour src/main/java/com/kaiounet/App.java}
\begin{lstlisting}[language=Java]
package com.kaiounet;

import javafx.application.Application;
import javafx.scene.Scene;
import javafx.scene.control.Button;
import javafx.scene.control.Label;
import javafx.scene.layout.VBox;
import javafx.stage.Stage;
import javafx.geometry.Pos;

public class App extends Application {
    @Override
    public void start(Stage primaryStage) {
        primaryStage.setTitle("Bonjour JavaFX");
        // ... (suite sur la diapositive suivante)
    }
    
    public static void main(String[] args) {
        launch(args);
    }
}
\end{lstlisting}
\end{block}
\end{frame}

\begin{frame}[fragile]{App.java - Code UI}
\begin{lstlisting}[language=Java]
@Override
public void start(Stage primaryStage) {
    primaryStage.setTitle("Bonjour JavaFX");
    
    Label label = new Label("Bienvenue dans JavaFX !");
    Button button = new Button("Cliquez-moi");
    
    button.setOnAction(e -> 
        label.setText("Bouton clique !"));
    
    VBox root = new VBox(10);
    root.setAlignment(Pos.CENTER);
    root.getChildren().addAll(label, button);
    
    Scene scene = new Scene(root, 400, 300);
    primaryStage.setScene(scene);
    primaryStage.show();
}
\end{lstlisting}
\end{frame}

\begin{frame}{Structure d'une Application JavaFX}
\begin{center}
\begin{tikzpicture}[node distance=1.5cm, auto]
    \tikzstyle{component} = [rectangle, rounded corners, minimum width=2.5cm, minimum height=0.8cm, text centered, draw=black, fill=blue!20, font=\small]
    \tikzstyle{arrow} = [thick,->,>=stealth]
    
    \node (app) [component] {Application};
    \node (stage) [component, below of=app] {Stage};
    \node (scene) [component, below of=stage] {Scene};
    \node (layout) [component, below of=scene] {Layout (VBox)};
    \node (nodes) [component, below of=layout] {Nodes (Button, Label)};
    
    \draw [arrow] (app) -- node[right, font=\tiny] {crée} (stage);
    \draw [arrow] (stage) -- node[right, font=\tiny] {contient} (scene);
    \draw [arrow] (scene) -- node[right, font=\tiny] {contient} (layout);
    \draw [arrow] (layout) -- node[right, font=\tiny] {contient} (nodes);
\end{tikzpicture}
\end{center}
\end{frame}

\section{Build et Exécution}

\begin{frame}[fragile]{Étape 5 : Builder le Projet}
\begin{block}{Compiler}
\begin{lstlisting}[style=bash]
mvn clean compile
\end{lstlisting}
\end{block}

\begin{alertblock}{Que Se Passe-t-il ?}
\begin{enumerate}
    \item Télécharge les dépendances JavaFX
    \item Compile les fichiers source Java
    \item Place les fichiers .class dans \texttt{target/classes}
\end{enumerate}
\end{alertblock}

\vspace{0.3cm}

\begin{exampleblock}{Premier Build}
Le premier build prend plus de temps en raison du téléchargement des dépendances. Les builds suivants sont plus rapides !
\end{exampleblock}
\end{frame}

\begin{frame}[fragile]{Étape 6 : Exécuter l'Application}
\begin{block}{Utilisation de Maven}
\begin{lstlisting}[style=bash]
mvn javafx:run
\end{lstlisting}
\end{block}

\vspace{0.5cm}

\begin{center}
\textit{(Votre fenêtre JavaFX devrait apparaître !)}
\end{center}
\end{frame}

\begin{frame}{Référence des Commandes Maven}
\begin{block}{Commandes Maven Courantes}
\begin{description}
    \item[\texttt{mvn clean}] Supprimer le répertoire \texttt{target/}
    \item[\texttt{mvn compile}] Compiler le code source
    \item[\texttt{mvn package}] Créer un fichier JAR
    \item[\texttt{mvn javafx:run}] Exécuter l'application JavaFX
    \item[\texttt{mvn clean compile}] Nettoyer et compiler
\end{description}
\end{block}

\begin{alertblock}{Astuce Pro}
Les commandes peuvent être enchaînées : \texttt{mvn clean compile javafx:run}
\end{alertblock}
\end{frame}

\section{Configuration de Make}

\begin{frame}[fragile]{Pourquoi Utiliser Make ?}
\begin{columns}
\column{0.45\textwidth}
\begin{block}{Sans Make}
\begin{lstlisting}[style=bash]
# A chaque fois...
mvn clean compile
mvn javafx:run
mvn package
\end{lstlisting}
\end{block}

\column{0.45\textwidth}
\begin{block}{Avec Make}
\begin{lstlisting}[style=bash]
# Beaucoup plus facile !
make compile
make run
make package
\end{lstlisting}
\end{block}
\end{columns}

\vspace{0.5cm}

\centering
\Large
Make = Raccourcis pour les commandes Maven !
\end{frame}

\begin{frame}[fragile]{Étape 7 : Créer le Makefile}
\begin{block}{Créer le Fichier}
\begin{lstlisting}[style=bash]
nvim Makefile
# ou
touch Makefile && code Makefile
\end{lstlisting}
\end{block}

\begin{alertblock}{Important}
\begin{itemize}
    \item Le fichier doit s'appeler exactement \texttt{Makefile}
    \item Utiliser des \textbf{TABulations} pour l'indentation (pas des espaces !)
    \item Placer à la racine du projet
\end{itemize}
\end{alertblock}
\end{frame}

\begin{frame}[fragile]{Makefile - Structure de Base}
\begin{lstlisting}[language=make]
.PHONY: help clean compile run

# Cible par defaut - affiche l'aide
help:
	@echo "Cibles disponibles :"
	@echo "  make compile - Compiler le projet"
	@echo "  make run     - Executer l'application"
	@echo "  make clean   - Nettoyer les artefacts"

# Compiler le projet
compile:
	mvn clean compile

# Executer l'application
run:
	mvn javafx:run

# Nettoyer les artefacts de build
clean:
	mvn clean
\end{lstlisting}
\end{frame}

\begin{frame}[fragile]{Makefile - Cibles Avancées}
\begin{lstlisting}[language=make]
# Empaqueter en JAR
package:
	mvn package

# Executer les tests
test:
	mvn test

# Installer les dependances
install:
	mvn install

# Rebuild complet et execution
rebuild: clean compile run

# Compilation et execution rapides
quick: compile run
\end{lstlisting}
\end{frame}

\begin{frame}[fragile]{Exemple Complet de Makefile}
\begin{lstlisting}[language=make, basicstyle=\ttfamily\tiny]
.PHONY: help clean compile run package test

help:
	@echo "Cibles disponibles :"
	@echo "  make compile  - Compiler le projet"
	@echo "  make run      - Executer l'application"
	@echo "  make package  - Empaqueter en JAR"
	@echo "  make clean    - Nettoyer les artefacts"
	@echo "  make test     - Executer les tests"

compile:
	mvn clean compile

run:
	mvn javafx:run

package:
	mvn package

clean:
	mvn clean

test:
	mvn test

rebuild: clean compile run
\end{lstlisting}
\end{frame}

\begin{frame}[fragile]{Étape 8 : Utiliser Votre Makefile}
\begin{block}{Afficher l'Aide}
\begin{lstlisting}[style=bash]
make
# ou
make help
\end{lstlisting}
\end{block}

\begin{block}{Utilisation Courante}
\begin{lstlisting}[style=bash]
# Compiler
make compile

# Executer l'application
make run

# Nettoyer et rebuilder
make rebuild
\end{lstlisting}
\end{block}
\end{frame}

\section{Revue de la Structure du Projet}

\begin{frame}[fragile]{Structure Finale du Projet}
\begin{block}{Votre Configuration Complète}
\begin{verbatim}
javafx-app/
|-- Makefile              # Automatisation Make
|-- pom.xml               # Configuration Maven
|-- src/
|   `-- main/
|       `-- java/
|           `-- com/
|               `-- kaiounet/
|                   `-- App.java
`-- target/              # Sortie du build (genere)
    `-- classes/
        `-- com/
            `-- kaiounet/
                `-- App.class
\end{verbatim}
\end{block}
\end{frame}

\begin{frame}{Fichiers Clés Expliqués}
\begin{description}
    \item[\texttt{pom.xml}] Configuration Maven - dépendances, plugins, propriétés
    \item[\texttt{Makefile}] Automatisation du build - raccourcis pour commandes courantes
    \item[\texttt{App.java}] Votre code d'application JavaFX
    \item[\texttt{target/}] Répertoire de sortie du build (ignoré par git)
\end{description}

\vspace{0.5cm}

\begin{alertblock}{Git Ignore}
Toujours ajouter \texttt{target/} à \texttt{.gitignore} !
\end{alertblock}
\end{frame}

\section{Workflow de Démarrage Rapide}

\begin{frame}[fragile]{Workflow de Développement}
\begin{block}{Cycle de Développement Typique}
\begin{lstlisting}[style=bash]
# 1. Modifier votre code
nvim src/main/java/com/kaiounet/App.java

# 2. Compiler
make compile

# 3. Executer
make run

# 4. Apporter des modifications, repeter 1-3
\end{lstlisting}
\end{block}

\begin{exampleblock}{Itération Rapide}
Pour un développement rapide : \texttt{make rebuild} (nettoie, compile et exécute)
\end{exampleblock}
\end{frame}

\begin{frame}{Problèmes Courants et Solutions}
\begin{block}{Problème : "mvn: command not found"}
\textbf{Solution :} Installer Maven ou ajouter au PATH
\end{block}

\begin{block}{Problème : "JavaFX components not found"}
\textbf{Solution :} Vérifier les dépendances \texttt{pom.xml}, exécuter \texttt{mvn clean compile}
\end{block}

\begin{block}{Problème : "Make: *** missing separator"}
\textbf{Solution :} Utiliser TAB (pas des espaces) pour l'indentation dans le Makefile
\end{block}

\begin{block}{Problème : Mauvaise mainClass}
\textbf{Solution :} Vérifier que \texttt{mainClass} dans \texttt{pom.xml} correspond à votre package
\end{block}
\end{frame}

\section{Prochaines Étapes}

\begin{frame}{Étendre Votre Projet}
\begin{block}{Ajouter Plus de Dépendances}
\begin{itemize}
    \item \texttt{javafx-web} - Composant WebView
    \item \texttt{javafx-media} - Lecture audio/vidéo
    \item \texttt{javafx-swing} - Intégration Swing
\end{itemize}
\end{block}

\begin{block}{Améliorer le Makefile}
\begin{itemize}
    \item Ajouter des cibles pour serveurs/clients spécifiques
    \item Créer des cibles de déploiement
    \item Ajouter des commandes de build docker
\end{itemize}
\end{block}

\begin{block}{Contrôle de Version}
\begin{itemize}
    \item Initialiser git : \texttt{git init}
    \item Ajouter \texttt{.gitignore}
    \item Committer votre configuration
\end{itemize}
\end{block}
\end{frame}

\begin{frame}{Meilleures Pratiques}
\begin{enumerate}
    \item \textbf{Garder pom.xml propre} - N'ajouter que les dépendances nécessaires
    \item \textbf{Utiliser les propriétés} - Définir les versions dans \texttt{<properties>}
    \item \textbf{Documenter le Makefile} - Ajouter des commentaires pour les cibles complexes
    \item \textbf{Tester régulièrement} - Ajouter des tests unitaires, exécuter avec \texttt{make test}
    \item \textbf{Contrôle de version} - Committer \texttt{pom.xml} et \texttt{Makefile}
\end{enumerate}

\vspace{0.5cm}

\begin{alertblock}{Ne Pas Committer}
Ne jamais committer le répertoire \texttt{target/} ou les fichiers spécifiques à l'IDE !
\end{alertblock}
\end{frame}

% \begin{frame}{Ressources}
% \begin{block}{Documentation}
% \begin{itemize}
%     \item JavaFX : \url{https://openjfx.io/}
%     \item Maven : \url{https://maven.apache.org/}
%     \item Make : \url{https://www.gnu.org/software/make/manual/}
% \end{itemize}
% \end{block}

% \begin{block}{Tutoriels}
% \begin{itemize}
%     \item Tutoriel JavaFX : \url{https://docs.oracle.com/javafx/}
%     \item Maven en 5 Minutes : \url{https://maven.apache.org/guides/}
% \end{itemize}
% \end{block}
% \end{frame}

\begin{frame}{Carte de Référence Rapide}
\begin{block}{Commandes Essentielles}
\begin{description}
    \item[\texttt{mvn archetype:generate}] Créer un nouveau projet
    \item[\texttt{make compile}] Compiler le code
    \item[\texttt{make run}] Exécuter l'application
    \item[\texttt{make clean}] Nettoyer le build
    \item[\texttt{make help}] Afficher les commandes disponibles
\end{description}
\end{block}

\begin{block}{Emplacements des Fichiers}
\begin{description}
    \item[Source] \texttt{src/main/java/}
    \item[Ressources] \texttt{src/main/resources/}
    \item[Tests] \texttt{src/test/java/}
    \item[Sortie] \texttt{target/}
\end{description}
\end{block}
\end{frame}

\begin{frame}{Résumé}
\begin{block}{Ce Que Nous Avons Couvert}
\begin{enumerate}
    \item Création de la structure du projet Maven
    \item Configuration des dépendances JavaFX
    \item Écriture d'une application JavaFX simple
    \item Build et exécution de l'application
    \item Configuration de Make pour l'automatisation
    \item Apprentissage du workflow de développement
\end{enumerate}
\end{block}

\vspace{0.5cm}

\begin{center}
\textbf{Vous êtes maintenant prêt à créer des applications JavaFX !}
\end{center}
\end{frame}

\end{document}
