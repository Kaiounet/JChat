\documentclass[12pt,a4paper]{report}

% === Packages ===
\usepackage[utf8]{inputenc}
\usepackage[T1]{fontenc}
\usepackage[french,english]{babel} % change or remove french if you only need english
\usepackage{graphicx}
\usepackage{hyperref}
\usepackage{listings}
\usepackage{xcolor}
\usepackage{geometry}
\usepackage{setspace}
\usepackage{caption}

% === Geometry and layout ===
\geometry{margin=1in}
\setstretch{1.3}

% === Code listing style ===
\lstdefinestyle{code}{
  backgroundcolor=\color{gray!10},
  basicstyle=\ttfamily\small,
  frame=single,
  breaklines=true,
  postbreak=\mbox{\textcolor{red}{$\hookrightarrow$}\space},
  keywordstyle=\color{blue},
  commentstyle=\color{gray},
  stringstyle=\color{teal},
  showstringspaces=false
}

% === Title Info ===
\title{
  \vspace{3cm}
  \textbf{Project Documentation}\\[1em]
  {\large Software Engineering Project}\\[2em]
  \includegraphics[width=0.3\textwidth]{example-image}\\[1em]
  {\small Your University / Organization}\\[3em]
}
\author{Kai \\[0.5em]\small Software Engineer}
\date{\today}

\begin{document}

% === Title Page ===
\maketitle
\thispagestyle{empty}
\newpage

% === Abstract ===
\begin{abstract}
This document provides an overview of the project’s purpose, design, and implementation details. 
It includes architectural decisions, technologies used, and results obtained.
\end{abstract}
\newpage

% === Table of Contents ===
\tableofcontents
\newpage

% === Introduction ===
\chapter{Introduction}
Explain the problem this project addresses, its motivation, and overall goals.
Include project scope and main objectives.

% === Requirements ===
\chapter{Requirements}
List functional and non-functional requirements.
\section{Functional Requirements}
\begin{itemize}
  \item The user can create an account.
  \item The user can submit and view projects.
\end{itemize}
\section{Non-functional Requirements}
\begin{itemize}
  \item The application must be responsive.
  \item Security must follow best practices.
\end{itemize}

% === Design & Architecture ===
\chapter{System Design}
Explain system architecture, components, and interactions.
You can include diagrams:
\begin{figure}[h]
  \centering
  \includegraphics[width=0.8\textwidth]{example-image}
  \caption{System Architecture Diagram}
\end{figure}

% === Implementation ===
\chapter{Implementation}
Show snippets of key code sections.
\begin{lstlisting}[style=code, language=Python, caption={Example API Endpoint}]
@app.route('/users', methods=['GET'])
def list_users():
    return jsonify([user.to_dict() for user in User.query.all()])
\end{lstlisting}

% === Results & Testing ===
\chapter{Testing and Results}
Describe testing strategies, results, and screenshots if needed.

% === Conclusion ===
\chapter{Conclusion}
Summarize the work, challenges faced, and possible future improvements.

% === References ===
\chapter*{References}
\addcontentsline{toc}{chapter}{References}
\begin{thebibliography}{9}
\bibitem{ref1} Author, \textit{Book Title}, Publisher, Year.
\bibitem{ref2} Website: \url{https://example.com}
\end{thebibliography}

\end{document}
